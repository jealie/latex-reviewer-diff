\documentclass{article}
\usepackage{reviewer} % loads the reviewer package

\usepackage{kantlipsum} % for blind text
\title{Genius paper that nobody understood}

\begin{document}

\begin{DIFnomarkup}

Dear Dr House,

We attach below the manuscript with our responses. The changes to the original manuscript are indicated by hyperlinked, numbered marks in the right-hand margin of the revised manuscript document below. Additions to the text are underlined in blue; deletions are indicated with red strike-through marks.

Sincerely, the authors

\startreview
\vspace{3cm}
REVIEWER \#1:

Can you explain this part a bit further, but without going into detail.

\us

We followed this reviewer's advice and updated the manuscript.
See changes at \reviewref{explaindetails}.

\them

\vspace{3cm}
REVIEWER \#2:

Not sure how to say this diplomatically, but the manuscript is really dull.

\us

We respectfully disagree with the reviewer's assessment of our work.
Nonetheless we updated the conclusion to make it look more interesting (\reviewref{a_paragraph}).

\stopreview\end{DIFnomarkup}

\maketitle

\kant[1]

\reviewlabel{explaindetails}The noumena have nothing to do with, thus, the Antinomies. What we
have alone been able to show is that the things in themselves constitute the
whole content of human reason, as is proven in the ontological manuals.

\kant[2]

\reviewlabel{a_paragraph}What we have alone been able to show is that our a posteriori concepts (and
it is obvious that this is the case) are what first give rise to the transcendental
unity of apperception. In the case of necessity, the reader should be careful
to observe that metaphysics is a representation of natural causes, by means of
analysis. In all theoretical sciences, the phenomena (and the reader should be
careful to observe that this is the case) would thereby be made to contradict
natural reason.

\end{document}
